\documentclass[letterpaper]{article} % Feel free to change this
\usepackage{graphicx}
\usepackage{caption}
\usepackage{rotating}
\usepackage{amsmath}
% comment out the next two lines if it doesn't compile for you
% or you could just install this package as well
\usepackage{vmargin}   
\setmarginsrb{1.0in}{1.0in}{1.0in}{1.0in}{12pt}{10mm}{12pt}{0mm} 
\begin{document}

\title{ECE559 Fall2018, Constituent Coder Interleaver Group\\
\begin{large}Group Design Report\end{large}}
\author{xxx}
% \today\\
\date{\today}%Nov. 26, 2018} % Change this to the date you are submitting
\maketitle
\vspace{12.0cm}


\newpage 
\section{3GPP LTE Advanced Wireless System}
    \subsection{Introduction}
        The next advancement on wireless communication was proposed by the 3rd Generation Partnership Program (3GPP). Formally a candidate 4G to ITU-T in 2009, it was standardized in March of 2011.
        The biggest benefit of LTE Advanced is the ability to take advantage of avdanced topology networks. The usage of optimized heterogeneous networks and utilizing macrocells with low power nodes such as pico and femto cells create a much better network compared to macrocells (wide area high power base stations that covers a large radius). LTE Advanced also introduces multicarrier so we can utilize wider bandwidth, up to 100 MHz, which will support high data rates.
        
        Our focus this semester lied on the coding, multiplexing and mapping to physical channels for E-UTRA section of the LTE Advanced specification as seen in 3GPP TS 36.212 V10.5.0. The sub-systems extracted from this document that our class worked on is as follows:
        \begin{itemize}
            \item Code block segmentation and CRC
            \item Convolutional encoder
            \item Turbo-code constituent encoder
            \item Turbo-code interleaver
            \item Sub-block interleaver and multiplexer 
        \end{itemize}
    \subsection{Internal Interleaver}
        Our group was responsible for the turbo coder internal interleaver. The purpose of this sub-system is to block interleave the data, frame quality indicator (CRC), and the reserve bits input to the turbo encoder. This interleaving happens through a index generator function defined in section 5.1.3.2.3 of the document. First we define an index generator function that selects the certain index of output bits from a certain input bit index. We will call this $\Pi(i)$:
        $$\Pi(i) = (f_1\cdot i + f_2\cdot i^2 )\mod K$$
        where $f_1$, $f_2$ are defined in Table 5.1.3-3 in the 3GPP document. These values depend on the block size $K$. In our case, $K$ could only be two different sizes 1056 and 6144. This leads us to the following values for $f_1$ and $f_2$:
        \renewcommand\arraystretch{1.2}
        \begin{center}
            \begin{tabular}{|c|c|c|}
            \hline
                $K$ & $f_1$ & $f_2$  \\ \hline
                1056 & 17 & 66 \\ \hline 
                6144 & 263 & 480 \\ \hline
            \end{tabular}
        \end{center}
        Using these values, we can generate the output bits using the following function:
        $$c'_i = c_{\Pi(i)} \quad\forall i \in \{0, 1, 2, ..., K\}$$
        where $c_i$ is the $i^{th}$ bit of the input sequence and $c'_i$ is the $i^{th}$ bit of the output sequence defined by the function $\Pi(i)$.
        
        This sub-module is part of the larger sub-system, turbo-code constituent encoder. The turbo encoder's purpose is to encode the incoming data, CRC, and the two reserved bits. During encoding, the output tail sequence is also added. The encoder generates the output based on the code rate. In our case, the code rate was 1/3. It uses two systematic, recursive, convolutional encoders connected in parallel with an interleaver for the second encoder. The two recursive convolutional codes are called the constituents codes of the turbo code. Since there is delay after the interleaver processes the data, the second encoder must account for this delay of the incoming data sequence.
\section{Description of the design}
    2.  Description of the design, including:
        (mostly yao)
         o  interfaces to other functions in the encoder stack
        (mostly yao)
         o  identification of the data path through the function and
		    how the data path is controlled by FSMs, shown using a
			block diagram of the design as implemented. 
         o  for data path and associated combinational logic (not FSMs), include
            the RTL view from Quartus, with a brief prose description;  this can
            be used to describe the block diagram mentioned in the
			previous bullet

		(mostly yao)
         o  FSM designs, with pictorial description (state transition diagrams)
            and text, clearly indicating what each state
            represents, how inputs determine the next state and Moore and Mealy
            outputs for each state, and also an indication of how the FSM is
            related to data flow through the function
\section{Static timing analysis}
(mostly cheng)
    3.  Static timing analysis
		 o  from the static timing analysis results, identify (a) worst-case slack
		 	against setup violations;  (b) maximum frequency at which the design
			can be clocked;  (c) the critical path corresponding to the worst-case
			slack.
		 o	based on the max frequency, comment on the max throughput that can
		 	be sustained by the function given the current design
\section{Simulation results}
(mostly yao)
	4.  Simulation results
         o  what was simulated, what were the test conditions, what is the
            significance of the test in the context of the subsystem function
         o  show on the plots of the results, by clear annotation on the plots,
            and in a prose discription, what the simulation results indicate with
            respect to operation and performance of the subsystem

\section{Hardware test results}
(mostly vinith)
    5.  Hardware test results
         o  what was run on the hardware, was it run at speed or "statically"?
         o  show appropriate indication of the test results (e.g. prints of
            logic analyzer displays), with clear annotation on the plots and
            accompanying prose decription
         o  what were the test conditions, what is the significance of the test
            in the context of the subsystem function
         o  relationship to simulation test results

\section{A1}
The circuit in Figure 6.8 was and simulated to find A1. Here're the simulation result:

\begin{minipage}{1.0\textwidth}
% \includegraphics[width=14cm, angle=0]{p2_v2.png}
\centering
\captionof{figure}{Log magnitude and phase of A1}
\centering
\end{minipage}
\bigskip
% From your plots, estimate t

\section{Closed-loop gain of non-inverting mode amplifier}

The non-inverting mode amplifier circuit was built and simulated to find closed-loop gain Af. Here're the simulation result:


\section{Design an inverting mode amplifier}

Here are the conditions required:
\begin{align*}
VS/RS&=-VO/RF\\
VO/VS&=-RF/RS=-4\\
\end{align*}




\end{document}